\documentclass{article}
\usepackage{mathtools}
\usepackage[utf8]{inputenc}
\usepackage{multirow}
\usepackage{tocloft}
\usepackage{graphicx}
\usepackage{algorithm}
\usepackage{algorithmic}
\usepackage{hyperref}


\floatname{algorithm}{Algorithme}

\renewcommand{\contentsname}{Table des matières}
\renewcommand{\tablename}{\Large Tableau}
\renewcommand{\cftsecfont}{\Large}
\renewcommand{\cftsubsecfont}{\large}


\begin{document}
\begin{titlepage}
    \begin{center}
        \vspace*{1cm}
        
        \Huge
        \textbf{INFO-F-203 - Rapport}
        
        \vspace{0.5cm}
        \LARGE
        Projet 1
        
        \vspace{1.5cm}
        
        \textbf{Yahya Bakkali\\}
        \Large
        Matricule : 445166\\
        
		\vspace{0.5cm}        
        
        \LARGE
        \textbf{Maxime Hauwaert\\}
        
		\Large        
        Matricule : 461714\\
        
        \vspace{0.8cm}

        Date : Novembre 2018
        
    \end{center}
\end{titlepage}

\setcounter{tocdepth}{3}
\tableofcontents
\newpage
\large
	
\section{Introduction générale}
Ce projet a pour but de mettre en pratique des concepts sur les graphes vus au cours d’algorithmique 2 pour une meilleure compréhension et maîtrise de ceux-ci.

\section{Sous-arbre de poids maximum}
	\subsection{Introduction}
		Dans ce problème nous manipulons des arbres constitués de nœuds ayant un poids. Le problème consiste à transformer un arbre $T=(V,E)$ en arbre ${T'}=({V'},{E'})$ de façon à maximiser la fonction
		$$w({V'})=\sum_{v\in{V'}} w(v)$$

	\subsection{Choix d'implémentation}

		Pour stocker l'arbre nous avons décidé d'utiliser une classe nommée "Tree". Chaque nœud a un nom, un poids et une liste de ses enfants. Nous avons décidé de ne pas modifier l'arbre de départ mais de créer une liste qui contiendra le nom de tous les nœuds qui forme le sous-arbre de poids maximum.
	
		\subsection{Algorithme}
		
		\begin{algorithm}[H]
		\caption{maxContribution}
		\begin{algorithmic}[1]
		\REQUIRE liste nœuds\_à\_désactiver

		\STATE poids\_total = nœud.poids
	
		\FOR{chaque enfant du nœud}
	
		\IF{enfant.maxContribution()$<=$ 0}
		\STATE Ajouter enfant à nœuds\_à\_désactiver

		\ELSE
		\STATE Ajouter enfant.maxContribution() à poids\_total
	
		\ENDIF	
		\ENDFOR

		\RETURN poids\_total

		\end{algorithmic}
		\end{algorithm}
	
		La complexité de cet algorithme est de $O(n)$ car il parcourt chaque nœud de l'arbre, $O(n)$ et à chaque nœud on additionne deux valeurs en $O(1)$, on compare deux valeurs en $O(1)$ et on ajoute à chaque fois\footnote{Au pire des cas} le nom du noeud dans une liste donc $O(1)$. Donc la complexité finale est de $O(n)$.
		
	\subsection{Arbres aléatoires}
	Cette génération aléatoire d'arbres a une très bonne distribution. Tous les arbres sont possibles. Il y a  de 1 à n nœuds qui composeront l'arbre, 'n' étant 15 dans ce projet. Chaque nœud choisira tout simplement de qui il veut être l'enfant parmi les nœuds déja placés.
	
\section{Les hypergraphes et hypertrees}

	
	\subsection{Introduction}
	Dans cette partie nous manipulons des hyper-graphes et des hyper-graphes duals. Les outils utiles pour travailler sur les hyper-graphes sont les graphes d’incidence et les graphes primals. On manipule plusieurs concepts comme les graphes acycliques, les hyper-arbres, les cliques maximales, la chordalité et la couverture exacte.
	
	\subsection{Choix d'implémentation}
	Pour stocker l'hyper-graphe nous avons décidé d'utiliser une classe nommée "Hypergraph". Chaque objet de cette classe a un ensemble de ses sommets, un dictionnaire d'hyper-arêtes, une matrice d'incidence, un dictionnaire de sommets liés et un graphe primal.
	Nous avons aussi décidé d'utiliser une classe "PrimalGraph" pour une meilleure visibilité de nos différents algorithmes. Chaque objet de cette classe a un ensemble de ses sommets, un dictionnaire de ses hyper-arêtes, le nombre de sommets qu'il a et une liste de ses arêtes.	
	
	\subsection{Algorithmes}
	
		\begin{algorithm}[H]
		\caption{find\_cliques}
		\begin{algorithmic}[1]
		\REQUIRE R : \{nœuds d'une clique maximale\}, P : \{nœuds possibles dans une clique maximale\}, X : \{nœuds exclus\}
		\IF{P et X sont vides}
		\IF{la clique R est de taille $>=$ 2}
		\STATE Ajouter R a la liste des cliques
		\ENDIF
		
		\ELSE
		
		\STATE pivot = élement aléatoire de l'ensemble $P \cup X $
		
		\FOR{chaque sommet S dans l'ensemble P $\setminus$ \{sommets liés au pivot\} }
		
		\STATE newP = P $\cap$ \{sommets liés à S\}
		\STATE newR = R $\cup$ \{S\}
		\STATE newX = X $\cap$ \{sommets liés à S\}
		\STATE find\_cliques(newP,newR,newX)
		
		\STATE P = P $\setminus$ \{S\}
		\STATE X = X $\cup$ \{S\}
		\ENDFOR		
		
		\ENDIF
	
	
		\end{algorithmic}
		\end{algorithm}

\cite{Algo1} Tout graphe à n sommets a au maximum $3^{n/3}$ cliques maximales, et le temps d'exécution le plus défavorable de l'algorithme de Bron-Kerbosch (avec une stratégie pivot qui minimise le nombre d'appels récursifs effectués à chaque étape) est $O(3^{n/3})$, correspondant à cette limite.


		\begin{algorithm}[H]
		\caption{is\_chordal}
		\begin{algorithmic}[1]		
		
		\STATE unnumbered = ensemble des sommets du graphe
		\STATE s = sommet choisi aléatoirement dans unnumbered
		\STATE unnumbered = unnumbered $\setminus$ \{s\}
		\STATE numbered  = \{s\}
		\WHILE{unnumbered $!=$ \{$\emptyset$\}}
			\STATE Vertex = le sommet de unumbered qui a le plus de connections aux sommets de numbered
			\STATE unnumbered = unnumbered - Vertex
			\STATE numbered = numbered + Vertex
			\STATE clique\_wanna\_be = \{sommets liés à Vertex\} $\cap$ numbered
			\STATE subGraph	= Un sous-graphe induit des sommets appartenant à clique\_wanna\_be
			\IF{le subGraph n'est pas complet}
				\RETURN False
			\ENDIF
		\ENDWHILE
		\RETURN True
		\end{algorithmic}
		\end{algorithm}

\cite{Algo2} Au début, nous créons l'ensemble des sommets du graphe, "unnumbered", $O(N)$, choisissons un sommet 
arbitraire d'unnumbered $O(N)$, enlevons ce sommet d'unnumbered $O(1)$,
créons l'ensemble des sommets, "numbered", contenant ce sommet $O(1)$.
Ensuite, tant qu'unnumbered n'est pas vide $O(N)$ nous chercherons
le sommet "Vertex" dans unnumbered qui a le plus de connexions aux sommets 
de numbered $O(S * N * W) = O((N/2)^2*N) = O(N^3/4)$ parce que la cardinalité 
de l'ensemble $\{S+W\}$ est toujours égale à N, après nous supprimons ce sommet d'unnumbered en 
l'ajoutant à numbered $O(1)$ puis nous créons un ensemble qui contiendra 
l'intersection entre numbered et A $O(\min(W,len(A)))$ et le 
sous-graphe induit a partir de cet ensemble $O(N*\min(W,len(A))*\min(V,len(A)))$.
Ce qui fait en final une complexite de
$O(N*\max(N^3/4,\min(W,len(A)),N*\min(W,len(A))*\min(N,len(A))))$\\\\
\noindent
A : ensemble des sommets liés au sommet "Vertex"\\
N : nombre de sommets du graphe\\
S : nombre de sommets dans "unnumbered"\\
W : nombre de sommets dans "numbered"
		
		\begin{algorithm}[H]
		\caption{Algorithm\_X}
		\begin{algorithmic}[1]
		\REQUIRE Matrice
		
		\STATE Faire une copie de la matrice et exécuter l'algorithme sur cette matrice		
		\STATE Choisir la colonne C contenant un minimum de 1
		\STATE L = L'ensemble des lignes tel que $Matrice_{l,c}$ = 1, $\forall l \in L$
		\FOR{chaque ligne l de L}
			\STATE columnslist = [] 
			\STATE rowslist = []
			\STATE Ajouter la ligne à la solution partielle
			\FOR{chaque colonne j de la matrice}
				\IF{$Matrice_{l,j} = 1$}
					\FOR{chaque ligne i de la matrice}
						\IF{$Matrice_{i,j} = 1$}
							\STATE Ajouter la ligne i à rowlist
						\ENDIF									
					\ENDFOR
					\STATE Ajouter la colonne j de la matrice à columnslist	
				\ENDIF
			\ENDFOR
		\STATE Supprimer les lignes et les colonnes de la matrice présentes dans rowslist et columnslist
		\IF{la matrice n'est pas vide}
			\IF{toutes les colonnes de la matrice ont au moins un 1}
				\STATE Répéter cet algorithme de façon récursive sur la matrice réduite
			\ENDIF			
		\ELSE
			\STATE Ajouter la solution à l'ensemble des solutions
		\ENDIF 
		\STATE Supprimer la ligne de la solution partielle
		\STATE Réutiliser la matrice de départ
		\ENDFOR
	
		\end{algorithmic}
		\end{algorithm}
		
\cite{Algo3}
Au début on fait une copie de la matrice sur laquelle on va appliquer l'algorithme $O(M*N)$, après on cherche les colonnes avec un minimum de 1 $O(M*N)$ ensuite on trouve les lignes $Matrice_{L,C}$ = 1 $O(N)$. Après pour chaque ligne des lignes trouvées précédemment on trouve les lignes et les colonnes à supprimer $O(M*N)$. On les supprime de la matrice $O(M*N)$. Si la matrice réduite n'est pas vide et que la colonne avec un minimum de 1 n'est pas 0, on répète cet algorithme sur cette matrice réduite de façon récursive ce qui donne une complexité non déterministe. Ça veut dire que l'algorihtme peut présenter des comportements différents sur des passages différents d'une matrice à l'autre réduite. Sinon la matrice est vide, on trouve une couverture exacte parmi les autres.


\begin{thebibliography}{9}

\bibitem{Algo1}
Bron–Kerbosch algorithm
\\\url{https://en.wikipedia.org/wiki/Bron%E2%80%93Kerbosch_algorithm}

\bibitem{Algo2}
networkx.algorithms.chordal
\\\url{https://networkx.github.io/documentation/stable/_modules/networkx/algorithms/chordal.html}

\bibitem{Algo3}
Knuth's Algorithm X
\\\url{https://en.wikipedia.org/wiki/Knuth%27s_Algorithm_X}
NP (complexity)
\\\url{https://en.wikipedia.org/wiki/NP_(complexity)}

\end{thebibliography}
		
	
	\subsection{Hypergraphes aléatoires}
	On crée d'abord l'ensemble de n sommets\footnote{15 étant une limite imposée}, n pouvant aller de 1 à 15, ensuite on détermine le nombre m d'hyper-arêtes\footnote{15 étant une limite arbitraire}, m pouvant aller de 0 à $\min(2^n-1,15)$.

\section{Interface graphique}
	Pour la première partie du projet, on affiche à l'écran tous les noeuds de l'arbre de départ. Les noeuds faisant partie du sous-arbre de poids maximum seront colorés en rouge et les autres seront colorés en gris.\\\ \\\ Pour la deuxième partie, on affiche à l'écran l'hypergraphe dual sous deux formes, son graphe primal et son graphe d'incidence, et on affiche si oui ou non c'est un hyper-arbre. (+ terminal)

\section{Librairies utilisées}

	\subsection{Numpy}
		C'est une librairie très utile dans ce projet pour l'utilisation d'opérations mathématiques telles que les fonctions sinus/cosinus, etc ainsi que dans la manipulation de l'aléatoire.

	\subsection{Matplotlib}
		C'est une librairie assez utile dans ce projet pour l'affichage d'objets mathématiques en 2D tels que des cercles, des lignes, etc.
	
	\subsection{Copy}
		C'est une librairie contenant la fonction "deepcopy" permettant de copier l'intégralité d'un objet sans qu'il n'y ait de liens entre l'ancien et le nouvel objet.

\section{Conclusion}
	En plus de la simple mise en pratique de certains concepts sur les graphes, ce projet nous a permis de développer nos compétences de travail en groupe.

\end{document}