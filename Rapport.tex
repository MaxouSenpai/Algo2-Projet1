\documentclass{article}
\usepackage{mathtools}
\usepackage[utf8]{inputenc}
\usepackage{multirow}
\usepackage{tocloft}
\usepackage{graphicx}
\usepackage{algorithm}
\usepackage{algorithmic}

\floatname{algorithm}{Algorithme}

\renewcommand{\contentsname}{Table des matières}
\renewcommand{\tablename}{\Large Tableau}
\renewcommand{\cftsecfont}{\LARGE}
\renewcommand{\cftsubsecfont}{\Large}


\begin{document}
\begin{titlepage}
    \begin{center}
        \vspace*{1cm}
        
        \Huge
        \textbf{INFO-F-203 - Rapport}
        
        \vspace{0.5cm}
        \LARGE
        Projet 1
        
        \vspace{1.5cm}
        
        \textbf{Yahya Bakkali\\}
        \Large
        Matricule : 445166\\
        
		\vspace{0.5cm}        
        
        \LARGE
        \textbf{Maxime Hauwaert\\}
        
		\Large        
        Matricule : 461714\\
        
        \vspace{0.8cm}

        Date : Novembre 2018
        
    \end{center}
\end{titlepage}

\setcounter{tocdepth}{3}
\tableofcontents
\newpage
\Large
	
\section{Introduction générale}
Ce projet a pour but de mettre en pratique des concepts vus au cours d’algorithmique 2.

\section{Sous-arbre de poids maximum}
	\subsection{Introduction}
		Le problème consiste à transformer un arbre $T=(V,E)$ en arbre ${T'}=({V'},{E'})$ de façon à maximiser la fonction
		$$w({V'})=\sum_{v\in{V'}} w(v)$$

	\subsection{Choix d'implémentation}
	
	\subsection{Algorithme}
	
	\begin{algorithm}
	\caption{maxContribution}	
	\begin{algorithmic}[1]
	\REQUIRE liste $nodes\_to\_delete$

	\STATE$poid\_total$ = node.weight
	
	\FOR{chaque $enfant$}
	
	\IF{maxContribution\_$enfant$ $<=$ 0}
	\STATE if

	\ELSE
	\STATE else
	\ENDIF


	
	\ENDFOR
	
	\end{algorithmic}
	\end{algorithm}

\section{Les hypergraphes et hypertrees}
	Ciao
	
\section{Librairies utilisées}

\subsection{Numpy}

\subsection{Matplotlib}

\section{Conclusion}
It's a conclusion

\end{document}